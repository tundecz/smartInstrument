\documentclass[../IoMusT.tex]{subfiles}
\begin{document}
\subsection{Sursele de motivație pentru realizarea lucrării}
\subsubsection{Internetul obiectelor muzicale}
Internetul obiectelor este o platformă vastă,  în plină dezvoltare, care a evoluat datorită convergenței a mai multor tehnologii. Din ce în ce mai multe obiecte de zi cu zi sunt combinate cu conectivitatea la internet astfel încât să poată colecta și schimba date între ele cu ajutorul unor senzori și actuatori. Scopul principal ale acestor creații este de a ne ușura și de a ne susține activitățile noastre zilnice.
\\
\par Internetul lucrurilor muzicale este derivat din acest domeniu, fiind o ramură relativ nouă și emergentă, mai mult bazată de cercetare și experimentare, având mai multe provocări în ceea ce privește latența. După cum se poate observa și din nume, rolul principal este jucat de domeniul muzicii, de instrumente și tot ceea ce ține de producție și recepție muzicală, fie că este vorba despre compoziție, de procesul de învățare sau doar de pură dis\-trac\-ți\-e. Întrucât majoritatea instrumentelor pot fi conectate la internet, există numeroase posibilități care facilitează atât publicul cât și interpretul pentru o performanță cât mai bună pe scenă și o interacțiune cât mai bună între audiență și interpreți.\cite{IoMusT}
\\
\par Sistemul senzorial joacă un rol decisiv în ceea ce privește experiența noastră la un concert, fie că este vorba despre muzică electrică sau clasică. Cu ajutorul internetului lucrurilor muzicale aceasta poate fi îmbunătățită și chiar poate veni în ajutor la persoanele cu dizabilități. Asta e și scopul principal al lucrării, de a crea un dispozitiv cu ajutorul căruia oamenii cu deficiență de auz să poată resimți vibrațiile muzicii astfel având și ei parte de o experiență mai bogată și o idee mai clară când vine vorba despre ascultarea muzica. De ce întocmai această temă? Pentru mine mu\-zi\-ca este și a fost mereu o parte foarte importantă din viața mea, eu cântând la pian de când mă știu. Datorită faptului că știam că vreau să fac ceva bazat pe IoT \footnote{Internet of Things}, marea provocare a fost de a găsi o modalitatea de a îmbina aceasta cu pasiunea mea astfel încăt să iasă ceva interesant și frumos. Am avut două direcții de orientare: să fac ceva pentru interpret sau pentru audiență.
\\
 \par Ideea de a face un dispozitiv pentru oamenii cu deficiență de auz nu este una originală însă nici una foarte populară nu e. S-au făcut experimente cu astfel de dispozitive pe piață cu rezultate pozitive, din ce se poate citi de pe internet, însă ideea de a folosi un instrument muzical ca input în loc de un fișier audio e o idee nouă, pe care nu am mai întâlnit-o până acum. Fiind un subiect atăt de rar abordat decizia mea a fost să experimentez și eu acest topic, alegând ca sursă de muzică un pian electric cu o interfață MIDI.

\subsubsection{Pe scurt despre muzica în comunitatea surdă}
Pentru noi este un lucru firesc și obișnuit de a aude sunetele din jurul nostru și mai ales de a asculta muzică fie pe telefon sau prin participarea la concerte. Într-o situație puțin mai neplăcută se află cei surzi, care cu toate că pot comunica prin limbajul semnelor, muzica pentru ei nu e ceva ce pot auzi, ci mai degrabă este ceva ce pot simți numai cu ajutorul vibrațiilor amplificate. În cazul lor celelalte simțuri, prin plasticitatea creierului, lucrează împreună pentru a compensa pierderea auzul \cite{DEAF}, simțul lor tactil astfel fiind mult mai dezvoltat. 
\\
\par O persoană cu deficiență de auz care cântă la un instrument muzical are o altă percepție asupra muzicii decât audiența. În cazul unui interpret canalul haptic este implicat într-o buclă complexă de acțiune. Muzicantul interacționează fizic cu instrumentul, pe de o parte, pentru a genera sunet, iar pe de altă parte, pentru a recepționa și percepe răspunsul fizic al instrumentului care este simțit sub forma unor vibrații \cite{Haptic}. În cazul audienței surde acest set de vibrații nu este resimțită sau este este resimțită cu o intensitate foarte slabă. Dezvoltarea unui dispozitiv cu ajutorul căruia s-ar simți aceste vibrații intensificate cu siguranță ar aduce o nouă percepție asupra muzicii pentru aceste persoane.

\subsection{Prezentarea capitolelor}
În primul capitol al acestei lucrări sunt prezentate aspecte generale des\-pre tot ce înseamnă internetul obiectelor muzicale în comunitatea surdă. Tot\-o\-da\-tă este și descrisă motivația alegerii acestei teme de licență, rezultatele lucrării cât și posibilele extinderi.
\\
\par Cel de al doilea capitol are ca scop informarea cititorului despre tehnologiile hardware folosite în realizarea lucrării. Sunt amănunțite detalii despre placa de dezvoltare RaspberryPi și funcționalitatea motoarelor de vibrații folosite la crearea dispozitivului.
\\
\par În capitolul al treailea sunt prezentate toate tehnologiile software folosite atât în crearea aplicației de pe placa de dezvoltare cât și cele folosite în programarea aplicației Android. Este rezervat și o secțiune unde se explică folosirea protocolului de comunicare MIDI\footnote{Musical Instrument Digital Interface} și mesajele transmise de la instrumentele muzicale.
\\
\par Capitolul al patrulea este rezervat pentru prezentarea aplicației. Sunt amănunțite detalii despre procesul de implementare atât a aplicației de pe placa de dezvoltare Rasperry Pi cât și a cele mobile.
\\
\par Capitolul cinci are ca scop prezentarea unei concluzii în ceee ce privește proiectarea și programarea acestui dispozitiv.
\\
\par La sfârșit se află materialele bibliografice folosite pentru realizarea do\-cu\-men\-ta\-ți\-ei.
\subsection{Rezultate}
În urma dezvoltării proiectului am reușit, cu ajutorul unor motoare, să redau sub forma unor vibrații notele muzicale cântate la pian în timp real cu o latență minimală. Vibrațiile resimțite denotă adâncimea notei respective. Cu cât o notă muzicală este mai adâncă cu atât vibrațiile sunt de o putere mai mare. Pe lângă acest dispozitiv s-a creat și o aplicația Android cu ajutorul căreia utilizatorul are posibilitatea de a opri sau de a modifica vibrațiile simțite. Tot cu ajutorul acestei aplicații se pot vizualiza vibrațiile și frecvențele simțite sub forma unor culori pentru o atribuire mai bună între vibrații și muzică. 
\subsection{Posibile extinderi}
Posibilitatea de extindere a acestei aplicații este destul de mare ținând cont că dispozitivul creat este doar un prototip și Interentul lucrurilor mu\-zi\-ca\-le are un potențial de creștere destul de mare, fiind un domeniu mai nou.
\\
\par Datorită faptului că acest proiect a fost făcut să redea vibrațiile unui singur instrument muzical, din motive clare, este firesc ca această funcționalitate să poate să fie extinsă pe mai multe instrumente, nu numai pian. Astfel utilizatorul n-ar simți numai numai vibrațiile pianului ci mai degrabă ar simți un cumul de vibrații de la mai multe instrumente. Astfel s-ar extinde și aplicația mobilă cu mai multe funcționalități și chiar s-ar putea adăuga opțiunea de realitate virtuală pentru o experiență vizuală mai bună.
\\
 \par În momentul de față este folosit un singur Raspberry Pi atât pentru culegrea datelor din pian cât și pentru punerea în funcțiune a motoarelor. Cazul idealar fi ca fiecare instrument să aibă propria lui placă, la fel ca și dispozitivul purtat de către audiență. Provocarea care trebuie rezolvată în acest caz este latența care se va inpune din cauza faptului că datele vor trebui să fie transmise de la un dispozitiv la altul, asta implicând o posibilă nesincronizare între muzica de pe scenă și vibrațiile simțite. 
\\
\par Din punct de vedere stilistic, dispozitivul, cu ajutorul unor motoare mai rezistente, se poate încorpora într-o haină vestimentară, făcându-l astfel mai ușor de folosit.
\end{document}