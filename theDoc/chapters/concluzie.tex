Dezvoltarea unei aplicații ce ține de muzică și vine în ajutorul persoanelor cu deficiență de auz este un subiect sensibil și poate fi puțin mai greu de abordat datorită faptului că e greu să știi cum percep persoanele respective aceste lucruri. În cazul acestui proiect, unde poate conta  sincronizarea între notele muzicale cântate și culorile, vibrațiile simțite, este important nu doar codul scris în spatele programului ci și arhitectura cu care au fost concepute instrumentele folosite. Datorită faptului că toată aplicația are la bază un instrument muzical, modul în care acesta funcționează poate fi un pas crucial în reușita dispozitivului dezvoltat. Pe parcurs am întâmpinat mai multe dificultăț atât pe partea aceasta cât și pe partea dezvoltării codului și mai ales pe partea de hardware, unde cu siguranță pot veni niște îmbunătățiri în ceea ce privește structura și design-ul acestuia. Datorită faptului că va\-lo\-ri\-le alese pentru reprezentate vibrațiilor și culorilor în această aplicație au fost instinctive, rezultatele finale sunt mai multi subiective. Chiar dacă haptica folosită în realizarea acestui proiect servește pentru o anumită categorie de oameni, ar putea fi și pentru noi o experiență încercarea unei asemenea dispozitiv.
\\
\par Cu siguranță se poate spune că internetul lucrurilor muzicale dă o altfel de percepție despre tot ce înseamnă muzică și conectarea acestuia la obiectele, dispozitivele ce le folosim în fiecare zi. Ne deschide o lume plină oportunități noi și ne oferă posibilitatea de a ne duce creativitatea la un alt nivel și de a încerca lucruri noi. Muzica poate juca un rol important în viețile unora datorită influenței pe care o aduce asupra noastră. Îmbinarea acestuia cu tehnologiile prezente și crearea unor aplicații bazate pe inteligență artificială, realitate augmentată sau tehnologii bazate pe haptici (cum e și în cazul nostru) ar duce la noi experiențe și provocări în domeniul informaticii.
