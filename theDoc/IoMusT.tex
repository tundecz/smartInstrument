\documentclass[a4paper, 12pt, titlepage]{article}
\usepackage{indentfirst}
\renewcommand*\contentsname{Cuprins}
\renewcommand\refname{}
\renewcommand{\figurename}{Figura}
\renewcommand{\listfigurename}{Lista de figuri}
\usepackage{subfiles}
\usepackage[hyphens]{url}
\usepackage[hidelinks,breaklinks]{hyperref}
\usepackage{siunitx}
\usepackage{graphicx}
\usepackage{float}
\usepackage{wrapfig}
\graphicspath{{./images/}}

\begin{document}

\begin{titlepage}
\begin{center}
\vspace*{-3.0cm}
Universitatea Transilvania din Brașov\\Facultatea de Matematică si Informatică
\\
\vspace{5.0cm}
\LARGE Lucrare de licență 
\\
\vspace{0.3cm}
Haptici și culori muzicale controlate prin aplicație Android
\\
\vspace{4.7cm}
\large Czoguly Tünde
\\
\vspace{0.5cm}
Coordonator științific \\  Lector Dr. Anca Vasilescu
\\
\vspace{6.0cm}
\normalsize Brașov\\2020
\end{center} 
\end{titlepage}

\tableofcontents

\newpage

\listoffigures

\newpage

\section{Introducere}
\subfile{chapters/introducere}

\newpage

\section{Tehnologii hardware folosite}
\subfile{chapters/hardware}

\newpage

\section{Tehnologii software folosite}
\subfile{chapters/software}

\newpage

\section{Prezentarea aplicației}
\subfile{chapters/aplicatie}

\newpage

\section{Concluzii}
\subfile{chapters/concluzie}
\newpage

\section{Bibliogragie si webografie}
\bibliography{db}
\bibliographystyle{plain}
\end{document}